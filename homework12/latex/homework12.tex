\documentclass{article}

\usepackage{amsmath, amssymb, amsfonts}
\usepackage{graphicx}
\usepackage{fancyvrb}

\newcommand{\inner}[2]{\left\langle #1, #2 \right\rangle}
\renewcommand{\vec}{\mathbf}

\author{Matthew Dupraz}
\title{Homework 10}

\begin{document}
\maketitle

\subsection*{(a)}
We will show that for every $k, l  \in \{0, 1, \dots, n-1\}$
it holds that
\begin{equation*}
	\sum_{j=0}^{n-1}\omega_n^{-kj}\omega_n^{lj} = 
	\begin{cases}
		0 & \text{ if } k \neq l\\
		n & \text{ if } k = l
	\end{cases}
\end{equation*}
Suppose $k \neq j$, then we can write
\begin{equation*}
	\sum_{j=0}^{n-1}\omega_n^{-kj}\omega_n^{lj} =
	\frac{1 - \omega_n^{-kn}\omega_n^{ln}}
		{1 - \omega_n^{-k}\omega_n^l} = 0
\end{equation*}
because $\omega_n^n = 1$. If $k = j$, then
\begin{equation*}
	\sum_{j=0}^{n-1}\omega_n^{-kj}\omega_n^{lj} =
	\sum_{j=0}^{n-1}1 = n
\end{equation*}
which is what we wanted to show.

\subsection*{(b)}

We have that
\begin{align*}
	\inner{\hat{x}}{\hat{y}} &= 
	\inner{\frac{1}{n}F_nx}{\frac{1}{n}F_n y}\\
	&= \frac{1}{n}\inner{x}{\left(\frac{1}{\sqrt{n}}F_n\right)^*
	\frac{1}{\sqrt{n}}F_n y}\\
	&= \frac{1}{n}\inner{x}{y}
\end{align*}
We used the fact that $\frac{1}{\sqrt{n}}F_n$ is unitary.
Hence $\inner{x}{y} = n\inner{\hat{x}}{\hat{y}}$
We deduce that 
\begin{equation*}
	\|x\|_2 = \sqrt{\inner{x}{x}} =
	\sqrt{n\inner{\hat{x}}{\hat{x}}}
	= \sqrt{n}\|\hat{x}\|_2
\end{equation*}
	
\subsection*{(c)}

We have that
\begin{align*}
	\hat{z_k} &= \frac{1}{n}\sum_{j=0}^{n-1}z_j\omega_n^{-kj}\\
	&= \frac{1}{n}\sum_{j=0}^{n-1}\sum_{i=0}^{n-1}
	x_i y_{j-i}\omega_n^{-kj}\\
	&= \frac{1}{n}\sum_{j=0}^{n-1}\sum_{i=0}^{n-1}
	x_i\omega_n^{-ki} y_{j-i}\omega_n^{-k(j - i)}\\
	&= \frac{1}{n}\sum_{i=0}^{n-1}x_i\omega_n^{-ki}
	\sum_{j=0}^{n-1}
	 y_{j-i}\omega_n^{-k(j - i)}\\
	&= \frac{1}{n}\sum_{i=0}^{n-1}x_i\omega_n^{-ki}
	\sum_{j=-i}^{n-1-i}
	 y_j\omega_n^{-kj}\\
	&= \frac{1}{n}\sum_{i=0}^{n-1}x_i\omega_n^{-ki}
	\left(\sum_{j=-i}^{-1} y_j\omega_n^{-kj} + 
	\sum_{j=0}^{n-1-i} y_j\omega_n^{-kj}\right)\\
	&= \frac{1}{n}\sum_{i=0}^{n-1}x_i\omega_n^{-ki}
	\left(\sum_{j=-i}^{-1} y_{j + n}\omega_n^{-k(j + n)} + 
	\sum_{j=0}^{n-1-i} y_j\omega_n^{-kj}\right)\\
	&= \frac{1}{n}\sum_{i=0}^{n-1}x_i\omega_n^{-ki}
	\left(\sum_{j=n-i}^{n-1} y_j\omega_n^{-kj} + 
	\sum_{j=0}^{n-1-i} y_j\omega_n^{-kj}\right)\\
	&= \frac{1}{n}\sum_{i=0}^{n-1}x_i\omega_n^{-ki}
	\sum_{j=0}^{n-1} y_j\omega_n^{-kj}\\
	&= n\hat{x}_k\hat{y}_k
\end{align*}
which is what we wanted to show.

\subsection*{(d)}

Denote $\tilde{\vec{x}}$ the value returned by \texttt{fft}
in \textsc{Matlab}, then since $\tilde{\vec{x}} = F_n\vec{x}$
and $\hat{\vec{x}} = \frac{1}{n}F_n\vec{x}$, it follows that
$\hat{\vec{x}} = \frac{1}{n}\tilde{\vec{x}}$.
In particular, the equation we obtain in (c) becomes
$\tilde{z}_k = \tilde{x}_k\tilde{y}_k$

In \textsc{Matlab}, I wrote the following implementation:
\begin{Verbatim}[frame=single,
	label=\textsc{Matlab} code - convolution.m]
function z=convolution(x, y)
	xhat = fft(x);
	yhat = fft(y);
	z = ifft(xhat .* yhat);
\end{Verbatim}
For $\vec{x} = (1, \dots, 10)^T$ and
$\vec{y} = (10, \dots, 1)^T$,
the method yields
\begin{equation*}
	\vec{z} = (340, 305, 280, 265, 260, 265, 280, 305, 340, 385)^T
\end{equation*}

\subsection*{(e)}

Let $\vec{x}, \vec{y} \in \mathbb{R}^n$, and let 
\begin{equation*}
	p_{\vec{x}}(t) = x_0 + x_1t + \dots + x_{n-1}t^{n-1},~~~
	p_{\vec{y}}(t) = y_0 + y_1t + \dots + y_{n-1}t^{n-1}
\end{equation*}
then
\begin{equation*}
	p_{\vec{x}}(t)p_{\vec{y}}(t)
	= \sum_{i=0}^{n-1}x_it^i\sum_{j=0}^{n-1}y_jt^j
	= \sum_{i=0}^{n-1}\sum_{j=0}^{n-1}x_iy_jt^{i+j}
\end{equation*}
Hence the coefficient of $t^k$ is
\begin{equation*}
	\sum_{\substack{i+j=k \\ i,j \leq n-1}}x_iy_j
	= \sum_{i = 0}^{2n-2}x_iy_{k-i}
	\mathbf{1}\{i \leq k, i\leq n-1\}
	= \sum_{i = 0}^{2n - 2}x_i'y_{k-i}'
	= (\vec{x'}*\vec{y'})_k
\end{equation*}
where we set $\vec{x'}, \vec{y'} \in \mathbb{R}^{2n-1}$
to be the vectors $\vec{x},\vec{y}$ padded with zeros.
It follows that
\begin{equation*}
	p_{\vec{x}}(t)p_{\vec{y}}(t) = p_{\vec{z}}(t),
\end{equation*}
where $\vec{z} = (\vec{x'} * \vec{y'})$.
Hence using (d), we can compute the value of $\vec{z}$ in
$O(n\log_2 n)$ complexity, since \texttt{fft} and
\texttt{ifft} are both of complexity $O(n\log_2 n)$.

In \textsc{Matlab} I wrote the following implementation, using
the method from (d):
\begin{Verbatim}[frame=single,
	label=\textsc{Matlab} code - poly\_product.m]
function z=poly_product(x, y)
	n = size(x)(1);
	w = zeros(n-1, 1);
	z = convolution([x; w], [y; w]);
\end{Verbatim}
For the polynomials
\begin{equation*}
	p_{\vec{x}}(t) = 1 + t + 2t^2 + 3t^3,~~~
	p_{\vec{y}}(t) = 3 + 2t + t^2 + t^3
\end{equation*}
the method yields
\begin{equation*}
	p_{\vec{z}}(t) = 3 + 5t + 9t^2 + 15t^3 + 9t^4 + 5t^5 + 3t^6
\end{equation*}
\end{document}
