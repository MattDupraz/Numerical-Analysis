\documentclass{article}

\usepackage{amsmath, amsfonts, amssymb}
\usepackage{fancyvrb}

\title{Homework 3}
\author{Matthew Dupraz}

\begin{document}

\maketitle

\subsection*{(a)}

\textbf{Code for myexp.m}:
\begin{Verbatim}[frame=single]
function sn = myexp(x)
% Initialize sum at s_0 = 1
n = 0;
sn = 1;
% Add next term in sum, as long as the value of next term
                                              is large enough
while sn == 0 || abs(x)^(n+1)/(factorial(n+1) * abs(sn))
                                                   > 10^(-16)
   n = n + 1;
   sn = sn + x^n/factorial(n);
end

\end{Verbatim}
\subsection*{(b)}

\textbf{Output from main.m (first half)}:
\begin{Verbatim}[frame=single]
myexp(-20) = -8.47432e-09
Relative error: 5.11144
---
myexp(-0.5) = 0.606531
Relative error: 1.83045e-16
\end{Verbatim}

While the calculated value for $\exp(-0.5)$ is very precise,
we can see that the calculated value for $\exp(-20)$ not only has a high
relative error, but is also negative, which is not even in the range of $\exp$.
We can explain this through cancellation - if we calculate a sum
$\sum_{i=0}^n a_i$, then we get some error for each operation we do, so the
calculated value of the sum would actually be
\begin{equation}
	(\cdots((a_0 + a_1)(1 + \delta_1) + a_2)(1+\delta_2)\cdots +
	a_n)(1+\delta_n),
\end{equation}
where $|\delta_i| \leq u \forall i \in \{1, \dots, n\}$.
We can rewrite this as 
\begin{equation}
	a_0\prod_{j=1}^{n}
	(1 + \delta_j) + \sum_{i=1}^{n}a_i\prod_{j=1}^{n + 1 - i}(1 + \delta_j).
\end{equation}
The absolute error we would then get is
\begin{equation}
	a_0\left(\prod_{j=1}^{n}
	(1 + \delta_j) - 1\right) + \sum_{i=1}^{n}a_i\left(\prod_{j=1}^{n + 1 - i}
	(1 + \delta_j) - 1\right).
\end{equation}
In our case we are computing $\exp(-20)$, so $a_i = x^i/i!$, in particular, for
example $a_{20} \approx -4.3\mathrm{e}{+7}$,
so this term could contribute an error which could be as large as for example
\begin{equation}
|a_{20}\times u| \approx 4.3\mathrm{e}{+7} 2^{-53}
\approx 9.5\mathrm{e}{-8} \approx 4.8\mathrm{e}{-9}
\end{equation}
This error is very large compared to the expected sum which should converge to
$\exp(-20) \approx 2\mathrm{e}{-9}$, so this explains why the calculated sum
has a very big relative error.

\subsection*{(c)}

\textbf{Output from main.m (second half)}:
\begin{Verbatim}[frame=single]
Calculating relative errors for myexp2(x):
x = -20: 1.043430033174047e-14
x = -18: 5.865766577052833e-15
x = -16: 5.527514594362839e-15
x = -14: 1.782629539594293e-15
x = -12: 3.584325437040343e-15
x = -10: 5.223999827135927e-15
x = -8: 2.747166956792859e-15
x = -6: 1.749593497392117e-15
x = -4: 1.515403081718387e-15
x = -2: 6.152625158459809e-16
\end{Verbatim}
\textbf{Code from main.m}:
\begin{Verbatim}[frame=single]
% Symbolic function, which calculates
	                      the relative error of a given
% method for computing exp(x) at some given point
rel_err = @(f, x) abs(f(x) - exp(x))/exp(x);

% Display the calculated value and relative error of myexp
fprintf('myexp(-20) = %g\n', myexp(-20));
fprintf('Relative error: %g\n', rel_err(@myexp, -20));

disp('---')

fprintf('myexp(-0.5) = %g\n', myexp(-0.5));
fprintf('Relative error: %g\n', rel_err(@myexp, -0.5));

% Display the relative errors of myexp2 at a collection of points
disp('---')
disp('Calculating relative errors for myexp2(x):')
for x=-20:2:-2
   fprintf('x = %d: %.16g\n', x, rel_err(@myexp2, x));
end
\end{Verbatim}
\textbf{Code from myexp2.m}:
\begin{Verbatim}[frame=single]
function y = myexp2(x)

% Compute m such that abs(x/2^m) <= 1
m = ceil(log2(abs(x)));
z = x/2^m;

% Calculate exp(x/2^m)
y = myexp(z);

% Square m times to obtain exp(x)
for i=1:m
   y = y^2;
end
\end{Verbatim}
\end{document}
