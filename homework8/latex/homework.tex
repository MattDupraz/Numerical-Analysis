\documentclass{article}

\usepackage{amsmath, amsfonts, amssymb}
\usepackage{graphicx}
\usepackage{fancyvrb}

\title{Homework 9}
\author{Matthew Dupraz}

\begin{document}

\maketitle

\subsection*{(a)}

Let $x_i = ih$ for $i \in \{0, \dots, N\}$ and $h = 1/N$.
We will apply the composite trapezoidal rule to approximate
the solution at each $x_i$:
\begin{align*}
	\int_0^1 k(x_i, y)u(y) dy &\approx
	Q^{(1)}_h[k(x_i, \cdot)u(\cdot)]\\
	&= \sum_{j=0}^{N-1}\frac{x_{j+1} - x_j}{2}[
	k(x_i, x_j)u(x_j) + k(x_i, x_{j+1}u(x_{j+1}))]\\
	&= \frac{h}{2}\sum_{j=0}^{N-1}[
	k(x_i, x_j)u(x_j) + k(x_i, x_{j+1}u(x_{j+1}))]\\
	&= \frac{h}{2}k(x_i, 0)u(0) + \frac{h}{2}k(x_i,1)u(1)
	+ h\sum_{j=1}^{N-1}k(x_i, x_j)u(x_j)
\end{align*}

Consider $\vec{f} = [f(x_0, \dots, f(x_N)]^T$, we'll find values
$\vec(u) = [u(x_0), \dots, u(x_N)]^T$ such that
\begin{equation*}
	Q_h^{(1)}[k(x_i, \cdot)u(\cdot)] = f(x_i) ~~
	\forall i \in \{0, \dots, N\}
\end{equation*}
If we let
\begin{equation*}
	A = \begin{bmatrix}
		\frac{h}{2}k(x_1, 0) & hk(x_1, x_1) & \cdots &
		hk(x_1, x_{N-1}) & \frac{h}{2} k(x_1, x_N)\\
		\vdots & \vdots & & \vdots & \vdots\\
		\frac{h}{2}k(x_N, 0) & hk(x_N, x_1) & \cdots &
		hk(x_N, x_{N-1}) & \frac{h}{2} k(x_N, x_N)\\
	\end{bmatrix},
\end{equation*}
then using the formula for the composite trapezoidal rule we found
above, this is equivalent to $A\vec{u} = \vec{f}$

\subsection*{(b)}
We can see that if we let
\begin{equation*}
	D =
	\begin{bmatrix}
		\frac{h}{2} & & & & \\
		 & h & & & \\
		 & & \ddots & & \\
		 & & & h & \\
		 & & & & \frac{h}{2}
	\end{bmatrix},
\end{equation*}
then $A = KD$. And so
\begin{equation*}
	A\vec{u} = \vec{f} \iff KD\vec{u} = \vec{f}
		\iff D\vec{u} = K^{-1}\vec{f}.
\end{equation*}
By definition of $f_z$ and the formula for the composite trapezoidal
rule we found in part (a), we have that
\begin{equation*}
	f_z = [k(z, x_0)~\cdots~k(z, x_N)]D\vec{u}
\end{equation*}
and so we conclude
\begin{equation*}
	f_z = [k(z, x_0)~\cdots~k(z, x_N)]K^{-1}\vec{f}
\end{equation*}

\subsection*{(c)}
Let $i \in \{0, \dots, N\}$. If $K$ is invertible, then
\begin{equation*}
	[k(x_i, x_0)~\cdots~k(x_i, x_N)]K^{-1} = \vec{e_i}
\end{equation*}
since $[k(x_i, x_0)~\cdots~k(x_i, x_N)]$ is the $i$-th column of
the matrix $K$.
Hence
\begin{equation*}
	f_{x_i} = [k(x_i, x_0)~\cdots~k(x_i, x_N)]K^{-1}\vec{f}
	= \vec{e_i}\cdot\vec{f} = f_i = f(x_i)
\end{equation*}

\subsection*{(f)}


\end{document}
